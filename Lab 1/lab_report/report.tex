%%%%%%%%%%%%%%%%%%%%%%%%%%%%%%%%%%%%%%%%%
% Organic Chemistry Lab Report
% Lab 1: Recrystallization & Melting Point Determination
%
% Professional Research Paper Format (ACS-inspired)
%
% License: CC BY-NC-SA 4.0
%%%%%%%%%%%%%%%%%%%%%%%%%%%%%%%%%%%%%%%%%

%----------------------------------------------------------------------------------------
%	DOCUMENT CLASS AND PACKAGES
%----------------------------------------------------------------------------------------

\documentclass[
	10pt,        % Smaller font for two-column
	twocolumn,   % Professional two-column layout
	letterpaper,
]{article}

% Page geometry (journal-style margins)
\usepackage[
	top=0.75in,
	bottom=1in,
	left=0.75in,
	right=0.75in,
	columnsep=0.25in, % Space between columns
	headheight=14pt,
]{geometry}

% Core packages
\usepackage[utf8]{inputenc}
\usepackage[T1]{fontenc}
\usepackage{mathpazo}         % Palatino font
\usepackage{microtype}        % Professional micro-typography (spacing, protrusion)
\usepackage{graphicx}
\usepackage{booktabs}
\usepackage{amsmath,amssymb}
\usepackage{float}

% Chemistry-specific packages
\usepackage[version=4]{mhchem}
\usepackage{siunitx}
\sisetup{
	separate-uncertainty=true,
	range-phrase=--,
	range-units=single,
	per-mode=fraction,
	fraction-function=\frac
}
\usepackage{chemfig}

% Bibliography
\usepackage[
	backend=biber,
	style=chem-acs,
	sorting=none,
]{biblatex}
\addbibresource{references.bib}

% Colors and hyperlinks
\usepackage{xcolor}
\usepackage{hyperref}
\hypersetup{
	colorlinks=true,
	linkcolor=black,
	citecolor=blue!70!black,
	urlcolor=blue!70!black,
}
\usepackage{xurl}

% Make all in-text citations a distinct color
\AtEveryCite{\color{blue!70!black}}

% Section styling (compact, bold, sans-serif headings)
\usepackage{titlesec}
\titleformat{\section}{\large\bfseries\sffamily}{\thesection.}{0.5em}{}
\titleformat{\subsection}{\normalsize\bfseries\sffamily}{\thesubsection.}{0.5em}{}
\titleformat{\subsubsection}{\normalsize\itshape}{\thesubsubsection.}{0.5em}{}
\titlespacing*{\section}{0pt}{1.5ex plus 0.5ex minus 0.2ex}{1ex plus 0.2ex}
\titlespacing*{\subsection}{0pt}{1.2ex plus 0.3ex minus 0.1ex}{0.8ex plus 0.1ex}
\titlespacing*{\subsubsection}{0pt}{1ex plus 0.2ex minus 0.1ex}{0.5ex plus 0.1ex}

% Headers and footers (minimal for journal style)
\usepackage{fancyhdr}
\pagestyle{fancy}
\fancyhf{}
\renewcommand{\headrulewidth}{0.4pt}
\fancyhead[L]{\small\textit{Recrystallization \& Melting Point}}
\fancyhead[R]{\small\textit{M. Li}}
\fancyfoot[C]{\thepage}

% Paragraph formatting (indented, no extra spacing - classic journal style)
\setlength{\parindent}{1em}
\setlength{\parskip}{0pt}

% Caption formatting
\usepackage{caption}
\captionsetup{
	font=small,
	labelfont=bf,
	justification=justified,
	singlelinecheck=false,
	skip=6pt,
}

% Balance columns on last page
\usepackage{balance}

%----------------------------------------------------------------------------------------
%	TITLE AND AUTHOR INFORMATION
%----------------------------------------------------------------------------------------

\newcommand{\reporttitle}{Recrystallization and Melting Point Determination of Benzoic Acid}
\newcommand{\reportauthor}{Matthew Li}
\newcommand{\reportpartner}{Tintin Ding}
\newcommand{\reportinstructor}{Mr. Osei-Mensah}
\newcommand{\reportcourse}{CL Organic Chemistry}
\newcommand{\reportsection}{B4}
\newcommand{\reportinstitution}{Loomis Chaffee High School}
\newcommand{\reportdate}{\today}

%----------------------------------------------------------------------------------------
%	DOCUMENT BODY
%----------------------------------------------------------------------------------------

\begin{document}

%----------------------------------------------------------------------------------------
%	TITLE BLOCK (spans both columns)
%----------------------------------------------------------------------------------------

\twocolumn[
	\begin{@twocolumnfalse}
		\centering
		\vspace{0.5cm}
		{\Large\bfseries \reporttitle \par}
		\vspace{0.4cm}
		{\normalsize \reportauthor\textsuperscript{*} and \reportpartner \par}
		\vspace{0.2cm}
		{\small\itshape \reportinstitution, \reportcourse\ (Section \reportsection) \par}
		{\small\itshape Instructor: \reportinstructor \quad $\cdot$ \quad \reportdate \par}
		\vspace{0.6cm}
		
		% Abstract within the full-width block
		\noindent\rule{\textwidth}{0.4pt}
		\vspace{0.3cm}
		\begin{minipage}{0.92\textwidth}
			\small
			\textbf{Abstract:}
			\textit{[Write a concise summary of the experiment. Include: (1) the purpose---to purify impure benzoic acid via recrystallization and determine its melting point; (2) brief methods---dissolved in hot water, cooled for crystallization, vacuum filtered, measured melting points; (3) key results---report the melting points of impure and pure samples, and percent recovery; (4) conclusions---comment on the success of purification based on melting point data.]}
		\end{minipage}
		\vspace{0.3cm}
		\noindent\rule{\textwidth}{0.4pt}
		\vspace{0.5cm}
	\end{@twocolumnfalse}
]



%----------------------------------------------------------------------------------------
%	INTRODUCTION (+6 points)
%----------------------------------------------------------------------------------------

\section{Introduction}

% INSTRUCTIONS (DELETE BEFORE SUBMISSION):
% - State the PROBLEM and REASON for this experiment
% - Discuss the TECHNIQUES used (recrystallization, melting point determination)
% - Review relevant BACKGROUND information and THEORY
% - Include citations from the literature where appropriate
% - Explain WHY recrystallization works (solubility vs. temperature)
% - Explain HOW melting point indicates purity

% TODO: Write your introduction here

% Paragraph 1: Purpose and significance


Organic compounds are often impure, after synthesized in the laboratory or isolated from natural sources. This contamination needs to be removed before 
those compounds can be applied and utilized in other applications. One of the most common and important methods of purification, for an organic chemist to know,
is \textbf{recrystallization}. It is essentially a technique to remove impurities from organic compounds that are solid at room temperature. Because the solubility of
a compound in a solvent generally increases with temperature, if we allow the solution containing the compound to cool slowly until the solution becomes saturated,
we can obtain relatively pure crystals because molecules in the crystals have a greater affinity for other molecules of the same type. Afterwards, the impurities
are left in the solution, and the pure compound is isolated and crystallized. 

\vspace{1em}

The purpose of this experiment is to purify impure benzoic acid (\ce{C6H5COOH}) that contains sugar, via recrystallization. Then, we will determine
the melting point of both the impure and pure benzoic acid samples to confirm the success of the process of recrystallization and purification. Because
the melting point of an impure compound—in this case, benzoic acid with sugar—will typically be a wider range at a higher temperature, and because the
purified compound will have a much narrower range at a lower temperature, we can use the melting point to assess the purity of the compound. 

\vspace{1em}

Benzoic acid is a white or colorless crystalline organic compound whose structure consists of a benzene ring (\ce{C6H6}) and a carboxyl substituent. 
The substance occurs naturally in many plants, and salts of benzoic acid are used as food preservatives. It is also an important element for the industrial
synthesis of many other organic compounds.\cite{wikipedia_benzoic_acid} 

\vspace{1em}

According to the PubChem Compound Summary for benzoic acid, the melting point of benzoic acid is around \SI{122.4}{\celsius} and its solubility in water
is around \SI{3.4}{\milli\gram\per\milli\liter} at \SI{25}{\celsius}~\cite{pubchem_benzoic_acid}.
The melting point of sugar (sucrose) is around \SI{185.5}{\celsius} and its solubility in water is around \SI{2.12e6}{\milli\gram\per\liter} at \SI{25}{\celsius}~\cite{pubchem_sucrose}.



%----------------------------------------------------------------------------------------
%	EXPERIMENTAL DETAILS (+10 points)
%----------------------------------------------------------------------------------------

\section{Experimental Details}

% INSTRUCTIONS (DELETE BEFORE SUBMISSION):
% - Describe procedures in enough detail that someone else could replicate your work
% - Include all materials, apparatus, and quantities used
% - Report actual masses and volumes (not just "about 1 g")
% - Describe special precautions or safety considerations
% - Use past tense and passive voice (e.g., "The sample was heated...")

\subsection{Materials and Apparatus}

% TODO: List your materials
\begin{itemize}
	\item Impure benzoic acid sample (mass: 1.018 g)
	\item Deionized (DI) water
	\item \SI{125}{\milli\liter} Erlenmeyer flask or beaker
	\item \SI{150}{\milli\liter} beaker
	\item Hot plate
	\item Ice bath
	\item Vacuum filtration apparatus (Büchner funnel, filter flask, vacuum trap)
	\item Filter paper (mass: 1.099 g)
	\item Melting point apparatus
	\item Melting point capillary tubes
	\item Analytical balance
	\item Weighing boat
\end{itemize}

\subsection{Procedure}

% TODO: Describe your procedure
\subsubsection{Day 1: Recrystallization}


\begin{enumerate}
	\item Record the melting point of impure benzoic acid
	\item Weigh out approximately 1.00 g of impure benzoic acid
	\item Heat DI water to near boiling
	\item Dissolve the impure benzoic acid in hot water (total volume added: \SI{45}{\milli\liter})
	\item Allow solution to cool slowly to room temperature
	\item Place in ice bath to enhance crystallization
	\item Collect crystals by vacuum filtration
	\item Allow crystals to dry overnight
\end{enumerate}

\subsubsection{Day 2: Analysis}


\begin{enumerate}
	\item Retrieve dried crystals from drying oven
	\item Record mass of purified benzoic acid
	\item Measure melting point of purified benzoic acid
\end{enumerate}

%----------------------------------------------------------------------------------------
%	RESULTS (+6 points)
%----------------------------------------------------------------------------------------

\section{Results}

% INSTRUCTIONS (DELETE BEFORE SUBMISSION):
% - Present your data clearly using tables and figures
% - Include all measured values with appropriate significant figures
% - Use proper units throughout
% - Label all tables and figures with descriptive captions
% - Reference all tables and figures in the text

\subsection{Experimental Data}

Data is presented in Table~\ref{tab:data} and Table~\ref{tab:melting-points}.
% TODO: Replace placeholder values with your actual data
\begin{table}[htbp]
	\centering
	\caption{Summary of experimental mass and melting point data for benzoic acid recrystallization.}
	\label{tab:data}
	\begin{tabular}{@{}lrl@{}}
		\toprule
		\textbf{Measurement} & \textbf{Value} & \textbf{Unit} \\
		\midrule
		Mass of impure benzoic acid & \textit{1.018} & \si{\gram} \\
		Volume of hot deionizedwater used & \textit{45} & \si{\milli\liter} \\
		Mass of filter paper & \textit{1.099} & \si{\gram} \\
		Mass of filter paper + crystals & \textit{1.686} & \si{\gram} \\
		Mass of purified benzoic acid & \textit{0.587} & \si{\gram} \\
		\bottomrule
	\end{tabular}
\end{table}

\begin{table}[htbp]
	\centering
	\caption{Melting point data for impure and purified benzoic acid samples.}
	\label{tab:melting-points}
	\resizebox{\columnwidth}{!}{%
	\begin{tabular}{@{}lcc@{}}
		\toprule
		\textbf{Sample} & \textbf{MP Range} & \textbf{Width} \\
		\midrule
		Impure & \textit{112--132} \si{\celsius} & \textit{20} \si{\celsius} \\
		Purified & \textit{122--123} \si{\celsius} & \textit{1} \si{\celsius} \\
		Literature\textsuperscript{*} & \SI{122.4}{\celsius} & --- \\
		\bottomrule
	\end{tabular}%
	}
	
	{\footnotesize\textsuperscript{*}Literature melting point from the PubChem Compound Summary for benzoic acid.\cite{pubchem_benzoic_acid}}
\end{table}

\subsection{Calculations}

% TODO: Show your percent recovery calculation
\subsubsection{Percent Recovery}

Before calculating the actual percent recovery from the results, we can first find the theoretical maximum of benzoic acid that can be recovered from the recrystallization. 
We can do this using the solubility of benzoic acid in water at \SI{25}{\celsius}, which is around \SI{3.4}{\milli\gram\per\milli\liter}. 

Thus, 

\begin{align*}
\text{Solubility} &= \SI{3.4}{\milli\gram\per\milli\liter} \\
\text{Volume of water used} &= \SI{45}{\milli\liter} \\
\text{Maximum mass lost to solubility} &= 3.4\,\frac{\si{\milli\gram}}{\si{\milli\liter}} \times 45\,\si{\milli\liter} \\
                                      &= 153\,\si{\milli\gram} \\
                                      &= 0.153\,\si{\gram}
\end{align*}

Therefore, the theoretical maximum recovery is:

\begin{align*}
\text{Theoretical max} &= 1.018\,\si{\gram} - 0.153\,\si{\gram} \\
						&= 0.865\,\si{\gram}
\end{align*}


The percent recovery of benzoic acid was calculated using the following equation:

\begin{equation}
	\%\ \text{Recovery} = \frac{m_{\text{pure}}}{m_{\text{impure}}} \times 100\%
	\label{eq:recovery}
\end{equation}


\[
\text{Percent Recovery} = \frac{\textit{0.587} \text{ g}}{\textit{1.000} \text{ g}} \times 100\% = \textit{58.7}\%
\]

%----------------------------------------------------------------------------------------
%	DISCUSSION (+18 points)
%----------------------------------------------------------------------------------------

\section{Discussion}

% INSTRUCTIONS (DELETE BEFORE SUBMISSION):
% - Interpret your results and relate them to the stated objectives
% - Discuss what your results MEAN
% - Compare your results to expected/literature values
% - You MUST address all 7 questions from the lab handout in this section
% - Use subsections to organize your discussion logically

\subsection{Structure and Intermolecular Forces of Benzoic Acid}

% Question 1 (+2 points): Draw the bond line structure for benzoic acid
\subsubsection{Structure of Benzoic Acid}


% Example using chemfig (you can replace with an image):
\begin{figure}[htbp]
	\centering
	% TODO: Replace with your own drawing or include an image file
	\chemfig{*6(-=-(-COOH)=-=)}
	\caption{Bond line structure of benzoic acid (\ce{C6H5COOH}).}
	\label{fig:benzoic-acid}
\end{figure}

% Question 2 (+2 points): Identify the types of IMFs present
\subsubsection{Intermolecular Forces in Benzoic Acid}

\begin{itemize}
	\item Hydrogen bonding (carboxylic acid group)
	\item Dipole-dipole interactions
	\item London dispersion forces
\end{itemize}

% Question 3 (+2 points): Explain why benzoic acid has a high MP in terms of IMFs
\subsubsection{Relationship Between IMFs and Melting Point}

With the presence of Hydrogen bonding of the O-H group, where the Hydrogen atom is partially positive and the Oxygen atom is partially negative, Benzoic acid can form relavively strong IMF with its peers, thereby holding molecules together and have a higher melting point. Also, the size of the benzoic acid, and the C=O group, which led to LDF and dipole-diople interactions, respectively, also plays a role in IMF and the high melting point.
\subsection{Analysis of Purification Results}

% Question 4 (+2 points): Calculate the percent recovery
\subsubsection{Percent Recovery Analysis}

The percent recovery of Benzoic acid from crystalization is 58.7 percent, which is a reasonable yield. 

% Question 5 (+3 points): Compare MP of impure vs pure benzoic acid
\subsubsection{Comparison of Melting Points}

The melting point of impure benzoic acid ranges from 112-132 Celsius, with a range of ~20 degrees. The melting point of pure benzoic acid, however, are from 122-123 Celsius, with a range of only 1 degree. This drastic decrease in the range of melting point is caused by the increased purity of the substance, where the impure sample has a larger range of melting point than the pure sample. The sugar, which is the impurity in the sample, contributes to the increasing range of the melting point, and by removing the sugars through recrystallization, the pure benzoic acid has a narrower range of MP.
Explain the effect of impurities on the melting point of a substance.

% Question 6 (+2 points): Comment on purity based on MP range
\subsubsection{Purity Assessment}
% TODO: Examine the melting point range for your purified benzoic acid. 
% Comment on the purity of your sample based on:
% - How narrow was the melting range? (A range < 1°C indicates high purity)
% - How close was the observed melting point to the literature value?
The melting range is $123.0\,^\circ\mathrm{C} - 122.0\,^\circ\mathrm{C} = 1.0\,^\circ\mathrm{C}$, which indicates high purity of benzoic acid. The observed melting point is very close to the literature value, which is \SI{122.3}{\celsius}, also suggesting that the compound is pure benzoic acid.

\subsection{Error Analysis}

% Question 7 (+5 points): Discuss sources of error
% TODO: Discuss possible sources of error in this experiment.
% For EACH error you identify:
%   1. Identify the error (be specific)
%   2. Explain how it affected your results (higher/lower recovery, MP, etc.)
%   3. Describe what you could do differently to minimize this error
%
% Consider errors from:
%   - Recrystallization procedure (e.g., cooling too quickly, not enough water, loss during transfer)
%   - Vacuum filtration (e.g., incomplete collection, washing losses)
%   - Melting point determination (e.g., heating rate, capillary tube packing)
	Factors that might decrease the yield is the solubility of benzoic acid in water at room temperature, which is 3.4 mg/mL. Since 45mL of deionized water is used to dissolve  benzoic acid, about 0.153 g of benzoic acid remains in the water during the recrystallization process. Also, another error that possibly occurs during the process of heating the benzoic acid, where the temperature used is too high that evaporates some benzoic acid into air, producing a pungant smell, also resulting a decrease in recovery rate.

%----------------------------------------------------------------------------------------
%	CONCLUSION & SUMMARY (+3 points)
%----------------------------------------------------------------------------------------

\section{Conclusion}

[write here]

%----------------------------------------------------------------------------------------
%	REFERENCES (+2 points)
%----------------------------------------------------------------------------------------

\section*{References}
\addcontentsline{toc}{section}{References}

% INSTRUCTIONS (DELETE BEFORE SUBMISSION):
% - Include all sources cited in your report
% - Use ACS citation format
% - Primary literature sources are preferred over websites
% - At minimum, cite your source for the literature melting point and solubility

% Option 1: Using BibLaTeX (recommended)
% Create a references.bib file in the same folder with your bibliography entries
% Then uncomment the following line:
\printbibliography[heading=none]

% Option 2: Manual references (if not using BibLaTeX)
% (Not used; references are handled via BibLaTeX.)
%\begin{enumerate}
%	\item National Center for Biotechnology Information. PubChem Compound Summary for CID 243, Benzoic acid. \url{https://pubchem.ncbi.nlm.nih.gov/compound/Benzoic-acid} (accessed [date]).
%	
%	\item \textit{[Add additional references as needed]}
%\end{enumerate}

%----------------------------------------------------------------------------------------
%	APPENDIX (Optional)
%----------------------------------------------------------------------------------------

% Uncomment if you need to include raw data, additional calculations, etc.
%\newpage
%\appendix
%\section{Raw Data}
%\textit{[Include any raw data sheets, additional calculations, or supplementary information here.]}

% Balance columns on the last page
\balance

\end{document}

