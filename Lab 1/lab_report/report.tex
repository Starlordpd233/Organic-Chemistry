%%%%%%%%%%%%%%%%%%%%%%%%%%%%%%%%%%%%%%%%%
% Organic Chemistry Lab Report
% Lab 1: Recrystallization & Melting Point Determination
%
% Professional Research Paper Format (ACS-inspired)
%
% License: CC BY-NC-SA 4.0
%%%%%%%%%%%%%%%%%%%%%%%%%%%%%%%%%%%%%%%%%

%----------------------------------------------------------------------------------------
%	DOCUMENT CLASS AND PACKAGES
%----------------------------------------------------------------------------------------

\documentclass[
	10pt,        % Smaller font for two-column
	twocolumn,   % Professional two-column layout
	letterpaper,
]{article}

% Page geometry (journal-style margins)
\usepackage[
	top=0.75in,
	bottom=1in,
	left=0.75in,
	right=0.75in,
	columnsep=0.25in, % Space between columns
	headheight=14pt,
]{geometry}

% Core packages
\usepackage[utf8]{inputenc}
\usepackage[T1]{fontenc}
\usepackage{mathpazo}         % Palatino font
\usepackage{microtype}        % Professional micro-typography (spacing, protrusion)
\usepackage{graphicx}
\usepackage{booktabs}
\usepackage{amsmath,amssymb}
\usepackage{float}

% Chemistry-specific packages
\usepackage[version=4]{mhchem}
\usepackage{siunitx}
\sisetup{
	separate-uncertainty=true,
	range-phrase=--,
	range-units=single
}
\usepackage{chemfig}

% Bibliography
\usepackage[
	backend=biber,
	style=chem-acs,
	sorting=none,
]{biblatex}
\addbibresource{references.bib}

% Hyperlinks
\usepackage{hyperref}
\hypersetup{
	colorlinks=true,
	linkcolor=black,
	citecolor=blue!70!black,
	urlcolor=blue!70!black,
}
\usepackage{xurl}

% Section styling (compact, bold, sans-serif headings)
\usepackage{titlesec}
\titleformat{\section}{\large\bfseries\sffamily}{\thesection.}{0.5em}{}
\titleformat{\subsection}{\normalsize\bfseries\sffamily}{\thesubsection.}{0.5em}{}
\titleformat{\subsubsection}{\normalsize\itshape}{\thesubsubsection.}{0.5em}{}
\titlespacing*{\section}{0pt}{1.5ex plus 0.5ex minus 0.2ex}{1ex plus 0.2ex}
\titlespacing*{\subsection}{0pt}{1.2ex plus 0.3ex minus 0.1ex}{0.8ex plus 0.1ex}
\titlespacing*{\subsubsection}{0pt}{1ex plus 0.2ex minus 0.1ex}{0.5ex plus 0.1ex}

% Headers and footers (minimal for journal style)
\usepackage{fancyhdr}
\pagestyle{fancy}
\fancyhf{}
\renewcommand{\headrulewidth}{0.4pt}
\fancyhead[L]{\small\textit{Recrystallization \& Melting Point}}
\fancyhead[R]{\small\textit{M. Li}}
\fancyfoot[C]{\thepage}

% Paragraph formatting (indented, no extra spacing - classic journal style)
\setlength{\parindent}{1em}
\setlength{\parskip}{0pt}

% Caption formatting
\usepackage{caption}
\captionsetup{
	font=small,
	labelfont=bf,
	justification=justified,
	singlelinecheck=false,
	skip=6pt,
}

% Balance columns on last page
\usepackage{balance}

%----------------------------------------------------------------------------------------
%	TITLE AND AUTHOR INFORMATION
%----------------------------------------------------------------------------------------

\newcommand{\reporttitle}{Recrystallization and Melting Point Determination of Benzoic Acid}
\newcommand{\reportauthor}{Matthew Li}
\newcommand{\reportpartner}{Tintin Ding}
\newcommand{\reportinstructor}{Mr. O}
\newcommand{\reportcourse}{CL Organic Chemistry}
\newcommand{\reportsection}{B4}
\newcommand{\reportinstitution}{Loomis Chaffee High School}
\newcommand{\reportdate}{\today}

%----------------------------------------------------------------------------------------
%	DOCUMENT BODY
%----------------------------------------------------------------------------------------

\begin{document}

%----------------------------------------------------------------------------------------
%	TITLE BLOCK (spans both columns)
%----------------------------------------------------------------------------------------

\twocolumn[
	\begin{@twocolumnfalse}
		\centering
		\vspace{0.5cm}
		{\Large\bfseries \reporttitle \par}
		\vspace{0.4cm}
		{\normalsize \reportauthor\textsuperscript{*} and \reportpartner \par}
		\vspace{0.2cm}
		{\small\itshape \reportinstitution, \reportcourse\ (Section \reportsection) \par}
		{\small\itshape Instructor: \reportinstructor \quad $\cdot$ \quad \reportdate \par}
		\vspace{0.6cm}
		
		% Abstract within the full-width block
		\noindent\rule{\textwidth}{0.4pt}
		\vspace{0.3cm}
		\begin{minipage}{0.92\textwidth}
			\small
			\textbf{Abstract:}
			\textit{[Write a concise summary of the experiment. Include: (1) the purpose---to purify impure benzoic acid via recrystallization and determine its melting point; (2) brief methods---dissolved in hot water, cooled for crystallization, vacuum filtered, measured melting points; (3) key results---report the melting points of impure and pure samples, and percent recovery; (4) conclusions---comment on the success of purification based on melting point data.]}
		\end{minipage}
		\vspace{0.3cm}
		\noindent\rule{\textwidth}{0.4pt}
		\vspace{0.5cm}
	\end{@twocolumnfalse}
]



%----------------------------------------------------------------------------------------
%	INTRODUCTION (+6 points)
%----------------------------------------------------------------------------------------

\section{Introduction}

% INSTRUCTIONS (DELETE BEFORE SUBMISSION):
% - State the PROBLEM and REASON for this experiment
% - Discuss the TECHNIQUES used (recrystallization, melting point determination)
% - Review relevant BACKGROUND information and THEORY
% - Include citations from the literature where appropriate
% - Explain WHY recrystallization works (solubility vs. temperature)
% - Explain HOW melting point indicates purity

% TODO: Write your introduction here

% Paragraph 1: Purpose and significance
\textit{[State the purpose of this experiment. Explain why purification of organic compounds is important in chemistry.]}

% Paragraph 2: Recrystallization theory
\textit{[Explain the theory behind recrystallization. Discuss how solubility changes with temperature and why impurities are left behind in solution.]}

% Paragraph 3: Melting point and purity
\textit{[Explain how melting point is used to assess purity. Discuss the relationship between impurities and melting point range/depression.]}

% Paragraph 4: Benzoic acid properties
\textit{[Provide background on benzoic acid (\ce{C6H5COOH}), including its structure, physical properties, and solubility in water. Cite your sources.]}

%----------------------------------------------------------------------------------------
%	EXPERIMENTAL DETAILS (+10 points)
%----------------------------------------------------------------------------------------

\section{Experimental Details}

% INSTRUCTIONS (DELETE BEFORE SUBMISSION):
% - Describe procedures in enough detail that someone else could replicate your work
% - Include all materials, apparatus, and quantities used
% - Report actual masses and volumes (not just "about 1 g")
% - Describe special precautions or safety considerations
% - Use past tense and passive voice (e.g., "The sample was heated...")

\subsection{Materials and Apparatus}

% TODO: List your materials
\begin{itemize}
	\item Impure benzoic acid sample (mass: 1.018 g)
	\item Deionized (DI) water
	\item \SI{125}{\milli\liter} Erlenmeyer flask or beaker
	\item \SI{150}{\milli\liter} beaker
	\item Hot plate
	\item Ice bath
	\item Vacuum filtration apparatus (Büchner funnel, filter flask, vacuum trap)
	\item Filter paper (mass: 1.099 g)
	\item Melting point apparatus
	\item Melting point capillary tubes
	\item Analytical balance
	\item Watch glass or weighing boat
\end{itemize}

\subsection{Procedure}

% TODO: Describe your procedure
\subsubsection{Day 1: Recrystallization}

\textit{[Describe the recrystallization procedure:
\begin{enumerate}
	\item Record the melting point of impure benzoic acid
	\item Weigh out approximately 1.00 g of impure benzoic acid
	\item Heat DI water to near boiling
	\item Dissolve the impure benzoic acid in hot water (record volume used)
	\item Allow solution to cool slowly to room temperature
	\item Place in ice bath to enhance crystallization
	\item Collect crystals by vacuum filtration
	\item Allow crystals to dry overnight
\end{enumerate}]}

\subsubsection{Day 2: Analysis}

\textit{[Describe the analysis procedure:
\begin{enumerate}
	\item Retrieve dried crystals from drying oven
	\item Record mass of purified benzoic acid
	\item Measure melting point of purified benzoic acid
\end{enumerate}]}

%----------------------------------------------------------------------------------------
%	RESULTS (+6 points)
%----------------------------------------------------------------------------------------

\section{Results}

% INSTRUCTIONS (DELETE BEFORE SUBMISSION):
% - Present your data clearly using tables and figures
% - Include all measured values with appropriate significant figures
% - Use proper units throughout
% - Label all tables and figures with descriptive captions
% - Reference all tables and figures in the text

\subsection{Experimental Data}

% TODO: Replace placeholder values with your actual data
\begin{table}[htbp]
	\centering
	\caption{Summary of experimental mass and melting point data for benzoic acid recrystallization.}
	\label{tab:data}
	\begin{tabular}{@{}lrl@{}}
		\toprule
		\textbf{Measurement} & \textbf{Value} & \textbf{Unit} \\
		\midrule
		Mass of impure benzoic acid & \textit{XX.XX} & \si{\gram} \\
		Volume of hot water used & \textit{XX} & \si{\milli\liter} \\
		Mass of filter paper & \textit{X.XXX} & \si{\gram} \\
		Mass of filter paper + crystals & \textit{X.XXX} & \si{\gram} \\
		Mass of purified benzoic acid & \textit{X.XXX} & \si{\gram} \\
		\bottomrule
	\end{tabular}
\end{table}

\begin{table}[htbp]
	\centering
	\caption{Melting point data for impure and purified benzoic acid samples.}
	\label{tab:melting-points}
	\resizebox{\columnwidth}{!}{%
	\begin{tabular}{@{}lcc@{}}
		\toprule
		\textbf{Sample} & \textbf{MP Range} & \textbf{Width} \\
		\midrule
		Impure & \textit{XX--XX} \si{\celsius} & \textit{X.X} \si{\celsius} \\
		Purified & \textit{XX--XX} \si{\celsius} & \textit{X.X} \si{\celsius} \\
		Literature\textsuperscript{*} & \SI{122.4}{\celsius} & --- \\
		\bottomrule
	\end{tabular}%
	}
	
	{\footnotesize\textsuperscript{*}Cite source for literature MP.}
\end{table}

\subsection{Calculations}

% TODO: Show your percent recovery calculation
\subsubsection{Percent Recovery}

The percent recovery of benzoic acid was calculated using the following equation:

\begin{equation}
	\%\ \text{Recovery} = \frac{m_{\text{pure}}}{m_{\text{impure}}} \times 100\%
	\label{eq:recovery}
\end{equation}

\textit{(Show your calculation with actual values:)}

\[
\text{Percent Recovery} = \frac{\textit{mass recovered} \text{ g}}{\textit{initial mass} \text{ g}} \times 100\% = \textit{XX.X}\%
\]

%----------------------------------------------------------------------------------------
%	DISCUSSION (+18 points)
%----------------------------------------------------------------------------------------

\section{Discussion}

% INSTRUCTIONS (DELETE BEFORE SUBMISSION):
% - Interpret your results and relate them to the stated objectives
% - Discuss what your results MEAN
% - Compare your results to expected/literature values
% - You MUST address all 7 questions from the lab handout in this section
% - Use subsections to organize your discussion logically

\subsection{Structure and Intermolecular Forces of Benzoic Acid}

% Question 1 (+2 points): Draw the bond line structure for benzoic acid
\subsubsection{Structure of Benzoic Acid}

\textit{[Draw the bond line structure of benzoic acid using ChemDraw or similar software, or use the chemfig package. Include the figure with a proper caption.]}

% Example using chemfig (you can replace with an image):
\begin{figure}[htbp]
	\centering
	% TODO: Replace with your own drawing or include an image file
	\chemfig{*6(-=-(-COOH)=-=)}
	\caption{Bond line structure of benzoic acid (\ce{C6H5COOH}).}
	\label{fig:benzoic-acid}
\end{figure}

% Question 2 (+2 points): Identify the types of IMFs present
\subsubsection{Intermolecular Forces in Benzoic Acid}

\textit{[Identify ALL types of intermolecular forces (IMFs) present in benzoic acid:
\begin{itemize}
	\item Hydrogen bonding (carboxylic acid group)
	\item Dipole-dipole interactions
	\item London dispersion forces
\end{itemize}
Explain the origin of each type of IMF based on the molecular structure.]}

% Question 3 (+2 points): Explain why benzoic acid has a high MP in terms of IMFs
\subsubsection{Relationship Between IMFs and Melting Point}

\textit{[Explain why benzoic acid has a relatively high melting point (\SI{122.4}{\celsius}) in terms of its intermolecular forces. Discuss how the strong hydrogen bonding between carboxylic acid groups, combined with the aromatic ring's dispersion forces, requires significant thermal energy to overcome.]}

\subsection{Analysis of Purification Results}

% Question 4 (+2 points): Calculate the percent recovery
\subsubsection{Percent Recovery Analysis}

\textit{[Discuss your calculated percent recovery from Section~\ref{eq:recovery}. Is this value reasonable? What factors affect percent recovery in recrystallization? Consider:
\begin{itemize}
	\item The solubility of benzoic acid in water at different temperatures
	\item The volume of water used
	\item Loss during transfer and filtration
\end{itemize}]}

% Question 5 (+3 points): Compare MP of impure vs pure benzoic acid
\subsubsection{Comparison of Melting Points}

\textit{[Compare the melting point of the impure benzoic acid to that of the purified sample (see Table~\ref{tab:melting-points}). Address:
\begin{itemize}
	\item Was the melting point of the impure sample lower than the pure sample?
	\item Was the melting range of the impure sample wider?
	\item Do these observations make sense based on melting point depression theory?
\end{itemize}
Explain the effect of impurities on the melting point of a substance.]}

% Question 6 (+2 points): Comment on purity based on MP range
\subsubsection{Purity Assessment}

% TODO: Examine the melting point range for your purified benzoic acid. 
% Comment on the purity of your sample based on:
% - How narrow was the melting range? (A range < 1°C indicates high purity)
% - How close was the observed melting point to the literature value?

\textit{Write your purity assessment here based on the criteria above.}

\subsection{Error Analysis}

% Question 7 (+5 points): Discuss sources of error
% TODO: Discuss possible sources of error in this experiment.
% For EACH error you identify:
%   1. Identify the error (be specific)
%   2. Explain how it affected your results (higher/lower recovery, MP, etc.)
%   3. Describe what you could do differently to minimize this error
%
% Consider errors from:
%   - Recrystallization procedure (e.g., cooling too quickly, not enough water, loss during transfer)
%   - Vacuum filtration (e.g., incomplete collection, washing losses)
%   - Melting point determination (e.g., heating rate, capillary tube packing)

\textit{Write your error analysis here, addressing at least 2--3 specific sources of error.}

%----------------------------------------------------------------------------------------
%	CONCLUSION & SUMMARY (+3 points)
%----------------------------------------------------------------------------------------

\section{Conclusion}

% INSTRUCTIONS (DELETE BEFORE SUBMISSION):
% - Summarize the main objectives and findings
% - State whether the objectives were achieved
% - Highlight the significance of your results
% - Keep this section brief (1 paragraph)

\textit{[Write a brief conclusion that:
\begin{itemize}
	\item Restates the purpose of the experiment
	\item Summarizes the key results (percent recovery, melting points)
	\item States whether the purification was successful based on melting point data
	\item Mentions any directions for improvement if the experiment were repeated
\end{itemize}]}

%----------------------------------------------------------------------------------------
%	REFERENCES (+2 points)
%----------------------------------------------------------------------------------------

\section*{References}
\addcontentsline{toc}{section}{References}

% INSTRUCTIONS (DELETE BEFORE SUBMISSION):
% - Include all sources cited in your report
% - Use ACS citation format
% - Primary literature sources are preferred over websites
% - At minimum, cite your source for the literature melting point and solubility

% Option 1: Using BibLaTeX (recommended)
% Create a references.bib file in the same folder with your bibliography entries
% Then uncomment the following line:
% \printbibliography[heading=none]

% Option 2: Manual references (if not using BibLaTeX)
% TODO: Replace with your actual references
\begin{enumerate}
	\item National Center for Biotechnology Information. PubChem Compound Summary for CID 243, Benzoic acid. \url{https://pubchem.ncbi.nlm.nih.gov/compound/Benzoic-acid} (accessed [date]).
	
	\item \textit{[Add additional references as needed]}
\end{enumerate}

%----------------------------------------------------------------------------------------
%	APPENDIX (Optional)
%----------------------------------------------------------------------------------------

% Uncomment if you need to include raw data, additional calculations, etc.
%\newpage
%\appendix
%\section{Raw Data}
%\textit{[Include any raw data sheets, additional calculations, or supplementary information here.]}

% Balance columns on the last page
\balance

\end{document}

